% IEEEAerospace2012.cls requires the following packages: times, rawfonts, oldfont, geometry
\documentclass[twocolumn,letterpaper]{IEEEAerospaceCLS}  % only supports two-column, letterpaper format

% The next line gives some packages you may find useful for your paper--these are not required though.
%\usepackage[]{graphicx,float,latexsym,amssymb,amsfonts,amsmath,amstext,times,psfig}
% NOTE: The .cls file is now compatible with amsmath!!!

\usepackage[]{graphicx}    % We use this package in this document
\newcommand{\ignore}[1]{}  % {} empty inside = %% comment
\graphicspath{{imgs/}}

\begin{document}
\title{An Example of the 2021 IEEE Aerospace Conference Paper Format Using a \LaTeX~Environment}

\author{%
Nicolas Mandel\\ 
Queensland University of Technology\\
Australian Centre for Robotic Vision\\
QUT Centre for Robotics\\
nicolasjohann.mandel@hdr.qut.edu.au
\and 
Sophie Taylor\\
Queenslad University of Technology\\
Nowhere, ZS 99999\\
jane.smith@nowhere.edu
%%%% IMPORTANT: Use the correct copyright information--IEEE, Crown, or U.S. government. %%%%%
\thanks{\footnotesize 978-1-7281-7436-5/21/$\$31.00$ \copyright2021 IEEE}              % This creates the copyright info that is the correct 2021 data.
%\thanks{{U.S. Government work not protected by U.S. copyright}}         % Use this copyright notice only if you are employed by the U.S. Government.
%\thanks{{978-1-7281-7436-5/21/$\$31.00$ \copyright2021 Crown}}          % Use this copyright notice only if you are employed by a crown government (e.g., Canada, UK, Australia).
%\thanks{{978-1-7281-7436-5/21/$\$31.00$ \copyright2021 European Union}}    % Use this copyright notice is you are employed by the European Union.
}



\maketitle

\thispagestyle{plain}
\pagestyle{plain}



\maketitle

\thispagestyle{plain}
\pagestyle{plain}

\begin{abstract}
Limitations in power, size and weight in UAVs have resulted in researchers exploring alternative navigation approaches to reduce the computational load onboard UAVs. In this work we present an ontology for semantic based navigation for UAVs. Contextual information is commonly used for navigation, with semantics at the frontier of contemporary robotic-based navigation research showing promising results. However, the connection between spatial and semantic information is not yet clear and formal definitions are limited in the literature. This paper aims to provide an ontology to link spatial and semantic relations through Ologs, an application of the mathematical field of category theory. Contemporary semantic concepts from natural language are fused with a spatial-semantic hierarchy in a mathematical framework. The framework is tested in simulations and its applicability verified with different scenarios. Results indicate that the defined spatial structure allows for improved UAV navigational capabilities. The presented ontology is mathematically sound and can be adapted to a number of semantic navigation use-cases in agriculture, search and rescue or industrial inspections.
\end{abstract} 


\tableofcontents

%%%%%%%%%%%%%%%%%%%%%%%%%%%%%%%%%%%%%%
\section{Introduction} \label{sec:Intro}
%%%%%%%%%%%%%%%%%%%%%%%%%%%%%%%%%%%%%%
Some schmeel here will have to introduce the main stuff about why this is useful and we consider it some great work. The usual stuff.\\
This file is a copy of the ''Example.tex'' file, which still contains all the info about which sections do what and so forth. I have just deleted them from this document to keep this as a working version.\\
The first section is a brief summary of current works in the literature that I have come across.~\cite{alirezaie_exploiting_2017}

\section{Literature Summary} \label{sec:Lit}


%%%%%%%%%%%%%%%%%%%%%%%%%%%%%%%%%%%%%%%%%%%%%%%%%%%%%%%%%%%%%%%%%%%%%%%%%%%%%%%%%%%%%%%%%%%%%%%%%
\appendices{}              % note there is no {} to put a title. Each appendix has its own title
%%%%%%%%%%%%%%%%%%%%%%%%%%%%%%%%%%%%%%%%%%%%%%%%%%%%%%%%%%%%%%%%%%%%%%%%%%%%%%%%%%%%%%%%%%%%%%%%%
% For a single appendix, use the \appendix{} keyword and do not use the \section command.

\section{More Information}        % first appendix
%%%%%%%%%%%%%%%%%%%%%%%%%%
This is the first appendix. 

\subsection{Comments}
If you have only one appendix, use the ``appendix'' keyword.

\subsection{More Comments}
Use section and subsection keywords as usual.

\section{Yet More Information}    % second appendix
%%%%%%%%%%%%%%%%%%%%%%%%%%%%%%
This is the second appendix.



%%%%%%%%%%%%%%%%%%%%%%%%%%%%%%%%%%%%%%%%%%%%%%%%%%%%%%%%%%%%%%%%%%%%%%%%%%%%%%%%%%%%%%%%%%%%%%%%%%%%%%
\acknowledgments
The authors thank the Office of Naval Research for funding this project.



%%%%%%%%%%%%%%%%%%%%%%%%%%%%%%%%%%%%%%%%%%%%%%%%%%%%%%%%%%%%%%%%%%%%%%%%%%%%%%%%%%%%%%%%%%%%%%%%%%%%%%
\bibliographystyle{IEEEtran}
\bibliography{references}


%%%%%%%%%%%%%%%%%%%%%%%%%%%%%%%%%%%%%%%%%%%%%%%%%%%%%%%%%%%%%%%%%%%%%%%%%%%%%%%%%%%%%%%%%%%%%%%%%%%%%%
\thebiography
%% This biostyle allows you to insert your photo size 1in X 1.25in
\begin{biographywithpic}
{Nicolas Mandel}{}
Blablabla
\end{biographywithpic} 

\begin{biographywithpic}
{Sophie Taylor}{}
Blablabla
\end{biographywithpic}




\end{document}
