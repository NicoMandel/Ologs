% IEEEAerospace2012.cls requires the following packages: times, rawfonts, oldfont, geometry
\documentclass[twocolumn,letterpaper]{IEEEAerospaceCLS}  % only supports two-column, letterpaper format

% The next line gives some packages you may find useful for your paper--these are not required though.
%\usepackage[]{graphicx,float,latexsym,amssymb,amsfonts,amsmath,amstext,times,psfig}
% NOTE: The .cls file is now compatible with amsmath!!!

\usepackage[]{graphicx}    % We use this package in this document
\usepackage{hyperref}
\hypersetup{
    colorlinks=true,
    linkcolor=black,
    filecolor=black,      
    urlcolor=black,
    citecolor=black,
    % draft,
}
\newcommand{\ignore}[1]{}  % {} empty inside = %% comment
\newcommand\todo[1]{\textbf{\textcolor{red}{#1}}}
\graphicspath{{imgs/}}

\begin{document}
\title{An ontology for UAV based  Semantic Navigation}

\author{%
Nicolas Mandel\\ 
Queensland University of Technology\\
Australian Centre for Robotic Vision\\
QUT Centre for Robotics\\
nicolasjohann.mandel@hdr.qut.edu.au
\and 
Michael Milford\\
Queenslad University of Technology\\
Nowhere, ZS 99999\\
jane.smith@nowhere.edu
\and
Felipe Gonzalez\\
Queenslad University of Technology\\
Nowhere, ZS 99999\\
jane.smith@nowhere.edu
%%%% IMPORTANT: Use the correct copyright information--IEEE, Crown, or U.S. government. %%%%%
\thanks{\footnotesize 978-1-7281-7436-5/21/$\$31.00$ \copyright2021 IEEE}              % This creates the copyright info that is the correct 2021 data.
%\thanks{{U.S. Government work not protected by U.S. copyright}}         % Use this copyright notice only if you are employed by the U.S. Government.
%\thanks{{978-1-7281-7436-5/21/$\$31.00$ \copyright2021 Crown}}          % Use this copyright notice only if you are employed by a crown government (e.g., Canada, UK, Australia).
%\thanks{{978-1-7281-7436-5/21/$\$31.00$ \copyright2021 European Union}}    % Use this copyright notice is you are employed by the European Union.
}



\maketitle

\thispagestyle{plain}
\pagestyle{plain}



\maketitle

\thispagestyle{plain}
\pagestyle{plain}

\begin{abstract}
Limitations in power, size and weight in UAVs have resulted in researchers exploring alternative navigation approaches to reduce the computational load onboard UAVs. In this work we present an ontology for semantic based navigation for UAVs. Contextual information is commonly used for navigation, with semantics at the frontier of contemporary robotic-based navigation research showing promising results. However, the connection between spatial and semantic information is not yet clear and formal definitions are limited in the literature. This paper aims to provide an ontology to link spatial and semantic relations through Ologs, an application of the mathematical field of category theory. Contemporary semantic concepts from natural language are fused with a spatial-semantic hierarchy in a mathematical framework. The framework is tested in simulations and its applicability verified with different scenarios. Results indicate that the defined spatial structure allows for improved UAV navigational capabilities. The presented ontology is mathematically sound and can be adapted to a number of semantic navigation use-cases in agriculture, search and rescue or industrial inspections.
\end{abstract} 


\tableofcontents

%%%%%%%%%%%%%%%%%%%%%%%%%%%%%%%%%%%%%%
\section{Introduction} \label{sec:Intro}
%%%%%%%%%%%%%%%%%%%%%%%%%%%%%%%%%%%%%%

\subsection{Problem Introduction}

\subsection{Why is this problem important}
Contemporary robotics places great emphasis on including semantic signals into the perception-understanding-control-pipeline. Semantic information has been included into the passive part of SLAM~\cite{cadena_past_2016,zhang_hierarchical_2019}, as well as into the active decision~\cite{koch_automatic_2019,alirezaie_exploiting_2017}. Kostavelis and Gasteratos define semantics as " [...] related to the study between signs and the things to which they refer, that is their meaning."~\cite{kostavelis_semantic_2015}. However, in terms of spatial occurrence, multiple levels of signals are possible, such as individual objects, regions or areas~\cite{kostavelis_semantic_2015}. These occurences cause novel problems for definining successful navigation, as highlighted by Anderson et al.~\cite{anderson_evaluation_2018}. Furthermore, recent advances in Computer Vision have produced a wide range of different sensors signals to be annotated with semantic information, such as pixel-level segmentation, object detection or scene classification~\cite{alom_history_2018}, which operate on different spatial levels.\\
The influence of these novel signals on the level of names on conventional robotics algorithms, such as filters and path planners, is an active field of research. Furthermore, UAVs represent a special case of robotics, which are impacted by unique viewing angles, as well as risk-averse requirements.
\todo{Open Issues are in Kostavelis and Gasteratos at the end}
\subsection{What do we propose for solving it}
In this paper we provide an overview of a selected subset of the literature on combining semantic signals with spatial information in robotics to illustrate the evolution of the field with novel algorithms information, specifically in the context of UAVs. First, we provide a brief overview of early approaches from the turn of the century. Second, we display a select subset of contemporary approaches, which include semantic information into novel algorithms. Third, we highlight approaches that incorporate semantic information into UAV pipelines.\\
In the second section we conduct numerical experiments on the well-defined problem of semantic mapping of a two-dimensional occupancy grid~\cite{gonzalez_unmanned_2016} and show how conventional parameters surrounding the quality of observations, overlap and flight speed influence a naive bayesian filter. We supplement the filter through an informed prior and demonstrate the effectiveness and shortcomings of the approach.\\
As a future prospect, we briefly point to approaches from related semantic fields, such as speech recognition and language processing, which hold the potential to enhance the inclusion of semantic information~\cite{lienou_semantic_2010}.

\section{Literature Summary} \label{sec:Lit}
The results of the literature overview are summarised in appendix~\todo{Appendix naming}. The following sections provide an overview of the approaches.

\subsection{Historic Literature} \label{subsec:LitHist}
The association of sensory inputs with symbolic systems and intricate representations has been discussed from a cognitive viewpoint by Harnad et al.~\cite{harnad_symbol_1990}.\\
Chown and colleagues~\cite{chown_prototypes_1995} researched the "what" and "where" subsystems of the human cognition and their learning over time. They captured the influence of landmark recognition, recreation of consistent Euclidean maps, as well as the importance of gateways, representing transitions between regions.\\
Kuipers~\cite{kuipers_spatial_2000} connected places, defined as 0D points, through 1D paths with 2D regions in topological maps. The proposed maps had a sensory and control level, a causal level, a topological level and a metrical level, all of which were connnected hierarchically.\\ Kuipers et al.~\cite{kuipers_local_2004} also conducted experiments where local metric room representations were linked through topological constraints.\\ 
Galindo et al.~\cite{galindo_robot_2008} connected a spatial hierarchical representation consisting of areas and objects with a terminological box and connected between these two in a deterministic manner to enable inference and path planning.\\
Borkowski et al~\cite{borkowski_towards_2010} assigned a hierarchical place taxomony based on architectural house representations to a metric map and performed path-planning within this map while considering semantic constraints.\\
Tenorth~\cite{tenorth_knowrob-map_2010} built a topological representation of abstract objects and locations and associates these in a hierarchical fashion to enhance task planning by incorporating the relations of objects.\\
Krishnan and Krishnan~\cite{krishnan_visual_2010} constructed a hierarchical semantical-topological and explored the semantic nodes before proceeding to the next one, making use of transition areas as mentioned in~\cite{chown_prototypes_1995}.
\subsection{Contemporary Literature}
Contemporary image processing techniques have accelerated semantic spatial research through significant improvements in reliability and robustness of classification and localisation~\cite{alom_history_2018}.\\
Suenderhauf et al.~\cite{sunderhauf_meaningful_2017} associated point-clouds in SLAM with labels propagated through a modern neural network.\\
Zhang et al.~\cite{zhang_hierarchical_2019} used a hierarchical topic model to improve the performance of a SLAM system by actively integrating the object association problem into a hierarchical dirichlet process.\\
Yang et al.~\cite{yang_visual_2018} used scene priors derived from vector representations of class names to enhance a reinforcement learning module which relies on visual cues in indoor environments.\\
Wu et al.~\cite{wu_learning_2018} used reinforcement learning to train an agent to recognise a room and infer whether another room type is directly accessible from this room using a bernoulli distribution.\\
Chaplot et al.~\cite{chaplot_object_2020} won the 2020 CVPR challenge to navigate to an object goal by learning a semantic-metric map from visual cues and training a policy to generate frontier-based goals. Their research also highlighted that other RL policies were outperformed by frontier-based methods and were unable to generalize to the real-world.\\ 
\subsection{UAV Literature}
UAV-based systems present different approaches to including semantic information. A substantial amount of literature is concerned with passive acquisition of semantic information.\\
Le Saux and Sanfourche~\cite{saux_rapid_2013}, Sheppard and Rahnemoonfar~\cite{sheppard_real-time_2017}, Christie et al.~\cite{christie_semantics_2016} and Kyrkou et al.~\cite{kyrkou_dronet:_2018}, classified top-down images taken from a UAV in real-time with respect to different classes and approaches. \\
Cavaliere et al.~\cite{cavaliere_towards_2016,cavaliere_towards_2    \item FOV - number of cells (is also number of cells in inclusion)
    \item Overlap018} associated ''tracks'' coming from images with ''places'' from geospatial information to describe relations between them in a relational manner, as detailed in~\cite{landsiedel_review_2017}. The individual items were structured hierarchically to distinguish between different type of tracks, such as vehicles and humans on the first level and cars, motorcycles and trucks on the second level.\\
Drouilly et al.~\cite{drouilly_semantic_2015} developed a metric to evaluate the semantic quality of a path through an environment depending on the distribution of objects and the observation quality.

Semantic information has also been actively employed in navigation algorithms to enhance their performance.\\
Maturana et al.~\cite{maturana_looking_2017} annotated a 2.5D map with detection of cars to approach these.\\
Maravall et al.~\cite{maravall_navigation_2017} used the image entropy to navigate through a topological map to semantically salient places.\\ 
Alirezaie~\cite{alirezaie_exploiting_2017} developed a RRT implementation that discards semantically inadmissible points through geospatial information associated with the waypoints and significantly improved planning time for longer distances.\\
Dang et al.~\cite{dang_autonomous_2018} employed an object detector on top of a voxel grid to detect humans and bicycles.\\
Koch et al.~\cite{koch_automatic_2019} respected semantic constraints during the path planning step for 3D reconstruction using UAVs while maintaining the quality of the reconstruction.
Toudeshki et al.~\cite{toudeshki_robust_2018} combined a visual teach-and-repeat approach with an object detector to funnel the UAV to follow a path.

\subsection{Literature on models from NLP}
Semantics as the study of signs and their meaning has a rich background in natural language processing, with a variety of tasks concerned with prediction and clustering. The following sections aims at illustrating a few examples.\\
In the field of speech processing, Hidden Markov Models (HMM) have long been the state-of-the-art~\cite{rabiner_tutorial_1989} for structuring temporal data and many contemporary probabilistic approaches in robotics rely on the markov assumption~\cite{thrun_probabilistic_2005}. These models have been extended to include hierarchical models, such as the Hierarchichal HMM~\cite{fine_hierarchical_1998}, which has been successfully adapted for other tasks, such as activity recognition with multiple sensors~\todo{Cite Oliver - layered -here}, trajecory recognition with wireless signals~\todo{cite the other layered paper here too - Li}, tennis sequences~\todo{find citation here} and gesture recognition for robotics~\todo{cite third layered here - aarno}. The PhD thesis of Kevin Murphy~\cite{murphy_dynamic_2002} ties hierarchical HMMs to Dynamic Bayesian Networks and exhibits a variety of different representations and forms.\\
Spatial data in natural language processing has been represented through documents, consisting of mixtures of topics, which generate words. Techniques such as probabilistic Latent Semantic Analysis (pLSA)~\cite{hofmann_probabilistic_1999} and the seminal work on Latent Dirichlet Allocation (LDA) by Blei et al.~\cite{blei_latent_2003} represented documents as mixtures of topic, which generate mixtures of words and learn topics in an unsupervised fashion from bag-of-words models. These approaches have been adapted to image-tasks. Monay et al.~\cite{monay_plsa-based_2004} annotated images by cross-training topic recognition on the images and the captions. Fei-Fei and Perona~\cite{fei-fei_bayesian_2005} adopted the generative model of the LDA for the task of unsupervised image annotation and Bosch et al.~\cite{bosch_scene_2006} the pLSA with image features.\\
Lienou et al.~\cite{lienou_semantic_2010} annotate patches in satellite images using a LDA. 
\section{Methodology} \label{sec:Met}
\subsection{Literature Overview}
Problem: "Gateways" as defined by Chown~\cite{chown_prototypes_1995}, used by Kuipers~\cite{kuipers_local_2004}, Krishnan and Krishnan~\cite{krishnan_visual_2010} and Wu et al.~\cite{wu_learning_2018} for successful implementations are difficult outdoors - where a room ends is quite easy (with a door/window), but where does the lawn end and the golf court begin?\\
Problem: RL are not capable of generalizing well~\cite{chaplot_object_2020} and also do not consistently outperform generalized path planning.\\
Problem: UAV viewpoints provide a hurdle for use of off-the-shelf NNs~\cite{richardwebster_psyphy:_2019}.\\
Problem: UAV path-planning is often done in a way to reliably cover large areas~\cite{vanegas_novel_2018}.\\
This research is directed towards the section on understanding and evaluating the semantic information of a UAV as demonstrated in the pipeline shown in Mandel et al.~\cite{mandel_towards_2020}.\\
Numerical simulations are run as a first step to test whether hierarchical models allow for improved performance on the task of mapping.
\subsection{Experiments} \label{subsec:MetEx}
\subsubsection{Task}
Blabla.
\begin{itemize}
    \item Mapping task - predefined map, reproduce
    \item Lawnmower pattern
    \item Idea: using the FOV for the next step $FOV_{t+1}$ to assign a better prior to cells that have yet been unobserved
\end{itemize}
We conduct numerical experiments to uncover the influences of certain aspects of the mapping pipeline. The map is an evenly spaced grid $MxM$ defined by a one-hot vector of the true identity, $x_{i,j} = \{0,..., 1, ..., 0\}_K$. The probability of making an observation $z$ is given by $p(z\mid~x)$, the estimated identity of a cell is given by Bayes' rule as:
\begin{equation}
    p(x \mid z) = \frac{p(z \mid x)~p(x)}{p(z)}
\end{equation}
for sequential updates and assumed independence of map cells, where the posterior of incremental updates is used as a new prior $p(x)$. The maximum number of observations is bound by the stride (proportional to the flight velocity and the desired overlap), as well as the field of view.\\
The probability of the true state given an observation can be evaluated through a test data set for any detection algorithm.
\subsubsection{Variations}
\begin{itemize}
    \item path planning factors for lawnmower pattern
    \begin{itemize}
        \item FOV - number of cells (is also number of cells in inclusion)
        \item Overlap
    \end{itemize}
    \item Recognition capability (for the classes)
    \item Maps:
    \begin{itemize}
        \item Normal
        \item Transpose
        \item Sparse Map (already written)
        \item Dense map (more forest-like)
    \end{itemize}
    \item Algorithms:
    \begin{itemize}
        \item Engineering - lambda as a mixture parameter of prior for unobserved cells 
        \item predicting the area that the UAV is in, calculating a p(t) from it.
    \end{itemize}
\end{itemize}
\subsection{Evaluation}\label{subsec:MetEval}
To evaluate the reproduction quality of the posterior map for each individual state we use the log-likelihood, for each cell defined by:
\begin{equation}
    H_{i,j} = - \sum_{k} p_{i,j,k}~ln(q_{i,j,k})
\end{equation}
With $p_i$ the true state and $q_i$ the estimated map state. Since $p_i$ is a one-hot vector, this equation evaluates the certainty over all map cells.
\section{Results} \label{sec:Results}
\subsection{Literature Overview}
The semantic information in robotics is included on various levels, especially in the contemporary literature, without explicit definition
\begin{enumerate}
    \item Area
    \item Region
    \item Object
\end{enumerate}
Semantics, as study of signs, has far-reaching roots in language. Computer scientists working with language have put large efforts into attempting to model spoken and written language. Spoken language has employed Hierarchical models of language to separate spoken words (through HHMMs). Natural Language processing has looked deep and hard into classifying collections of words into topics and documents (which are considered mixtures of topics) for classification and proposition. 
example of successful application are Fei-Fei-Lis LDA Application, and the other ones for the scenes and images. And the one with the satellite images. Maybe we can use these to turn into sophisticated models?
What we are currently missing from these are the reliable amounts of ground-truth data.
In speech applications: Various forms of Hierarchical Hidden Markov Models, mostly elucidated by Kevin Murphy~\cite{murphy_dynamic_2002}
\subsection{Simulations}

\section{Discussion} \label{sec:Disc}
Why the stuff we do in this case might not be fully mathematically sound. But why we believe this may make semantic mapping approaches better by providing an interpretable framework with values to update 
\section{Future Work} \label{sec:Fut}
%%%%%%%%%%%%%%%%%%%%%%%%%%%%%%%%%%%%%%%%%%%%%%%%%%%%%%%%%%%%%%%%%%%%%%%%%%%%%%%%%%%%%%%%%%%%%%%%%
\appendices{}              % note there is no {} to put a title. Each appendix has its own title
%%%%%%%%%%%%%%%%%%%%%%%%%%%%%%%%%%%%%%%%%%%%%%%%%%%%%%%%%%%%%%%%%%%%%%%%%%%%%%%%%%%%%%%%%%%%%%%%%
% For a single appendix, use the \appendix{} keyword and do not use the \section command.

\section{More Information}        % first appendix
%%%%%%%%%%%%%%%%%%%%%%%%%%
This is the first appendix. 

\subsection{Comments}
If you have only one appendix, use the ``appendix'' keyword.

\subsection{More Comments}
Use section and subsection keywords as usual.

\section{Yet More Information}    % second appendix
%%%%%%%%%%%%%%%%%%%%%%%%%%%%%%
This is the second appendix.



%%%%%%%%%%%%%%%%%%%%%%%%%%%%%%%%%%%%%%%%%%%%%%%%%%%%%%%%%%%%%%%%%%%%%%%%%%%%%%%%%%%%%%%%%%%%%%%%%%%%%%
\acknowledgments
The authors thank the Office of Naval Research for funding this project.



%%%%%%%%%%%%%%%%%%%%%%%%%%%%%%%%%%%%%%%%%%%%%%%%%%%%%%%%%%%%%%%%%%%%%%%%%%%%%%%%%%%%%%%%%%%%%%%%%%%%%%
\bibliographystyle{IEEEtran}
\bibliography{references}


%%%%%%%%%%%%%%%%%%%%%%%%%%%%%%%%%%%%%%%%%%%%%%%%%%%%%%%%%%%%%%%%%%%%%%%%%%%%%%%%%%%%%%%%%%%%%%%%%%%%%%
\thebiography
%% This biostyle allows you to insert your photo size 1in X 1.25in
\begin{biographywithpic}{Nicolas Mandel}{blankpic.eps}
Blablabla
\end{biographywithpic} 

\begin{biographywithpic}{Michael}{blankpic.eps}
Blablabla
\end{biographywithpic}




\end{document}
