% IEEEAerospace2012.cls requires the following packages: times, rawfonts, oldfont, geometry
\documentclass[twocolumn,letterpaper]{IEEEAerospaceCLS}  % only supports two-column, letterpaper format

% The next line gives some packages you may find useful for your paper--these are not required though.
%\usepackage[]{graphicx,float,latexsym,amssymb,amsfonts,amsmath,amstext,times,psfig}
% NOTE: The .cls file is now compatible with amsmath!!!

\usepackage[]{graphicx}    % We use this package in this document
\usepackage{hyperref}
\hypersetup{
    colorlinks=true,
    linkcolor=black,
    filecolor=black,      
    urlcolor=black,
    citecolor=black,
    % draft,
}
\newcommand{\ignore}[1]{}  % {} empty inside = %% comment
\newcommand\todo[1]{\textbf{\textcolor{red}{#1}}}
\graphicspath{{imgs/}}

\begin{document}
\title{An ontology for UAV based  Semantic Navigation}

\author{%
Nicolas Mandel\\ 
Queensland University of Technology\\
Australian Centre for Robotic Vision\\
QUT Centre for Robotics\\
nicolasjohann.mandel@hdr.qut.edu.au
\and 
Michael Milford\\
Queenslad University of Technology\\
Nowhere, ZS 99999\\
jane.smith@nowhere.edu
\and
Felipe Gonzalez\\
Queenslad University of Technology\\
Nowhere, ZS 99999\\
jane.smith@nowhere.edu
%%%% IMPORTANT: Use the correct copyright information--IEEE, Crown, or U.S. government. %%%%%
\thanks{\footnotesize 978-1-7281-7436-5/21/$\$31.00$ \copyright2021 IEEE}              % This creates the copyright info that is the correct 2021 data.
%\thanks{{U.S. Government work not protected by U.S. copyright}}         % Use this copyright notice only if you are employed by the U.S. Government.
%\thanks{{978-1-7281-7436-5/21/$\$31.00$ \copyright2021 Crown}}          % Use this copyright notice only if you are employed by a crown government (e.g., Canada, UK, Australia).
%\thanks{{978-1-7281-7436-5/21/$\$31.00$ \copyright2021 European Union}}    % Use this copyright notice is you are employed by the European Union.
}



\maketitle

\thispagestyle{plain}
\pagestyle{plain}



\maketitle

\thispagestyle{plain}
\pagestyle{plain}

\begin{abstract}
Limitations in power, size and weight in UAVs have resulted in researchers exploring alternative navigation approaches to reduce the computational load onboard UAVs. In this work we present an ontology for semantic based navigation for UAVs. Contextual information is commonly used for navigation, with semantics at the frontier of contemporary robotic-based navigation research showing promising results. However, the connection between spatial and semantic information is not yet clear and formal definitions are limited in the literature. This paper aims to provide an ontology to link spatial and semantic relations through Ologs, an application of the mathematical field of category theory. Contemporary semantic concepts from natural language are fused with a spatial-semantic hierarchy in a mathematical framework. The framework is tested in simulations and its applicability verified with different scenarios. Results indicate that the defined spatial structure allows for improved UAV navigational capabilities. The presented ontology is mathematically sound and can be adapted to a number of semantic navigation use-cases in agriculture, search and rescue or industrial inspections.
\end{abstract} 


\tableofcontents

%%%%%%%%%%%%%%%%%%%%%%%%%%%%%%%%%%%%%%
\section{Introduction} \label{sec:Intro}
%%%%%%%%%%%%%%%%%%%%%%%%%%%%%%%%%%%%%%

\subsection{Problem Introduction}

\subsection{Why is this problem important}
Contemporary robotics places great emphasis on including semantic signals into the perception-understanding-control-pipeline. Semantic information has been included into the passive part of SLAM~\cite{cadena_past_2016,zhang_hierarchical_2019}, as well as into the active decision~\cite{koch_automatic_2019,alirezaie_exploiting_2017}. Kostavelis and Gasteratos define semantics as " [...] related to the study between signs and the things to which they refer, that is their meaning."~\cite{kostavelis_semantic_2015}. However, in terms of spatial occurrence, multiple levels of signals are possible, such as individual objects, regions or areas~\cite{kostavelis_semantic_2015}. These occurences cause novel problems for definining successful navigation, as highlighted by Anderson et al.~\cite{anderson_evaluation_2018}. Furthermore, recent advances in Computer Vision have produced a wide range of different sensors signals to be annotated with semantic information, such as pixel-level segmentation, object detection or scene classification~\cite{alom_history_2018}, which operate on different spatial levels.\\
The influence of these novel signals on the level of names on conventional robotics algorithms, such as filters and path planners, is an active field of research. Furthermore, UAVs represent a special case of robotics, which are impacted by unique viewing angles, as well as risk-averse requirements.
\subsection{What do we propose for solving it}
In this paper we provide an overview of a selected subset of the literature on combining semantic signals with spatial information in robotics to illustrate the evolution of the field with novel algorithms information, specifically in the context of UAVs. First, we provide a brief overview of early approaches from the turn of the century. Second, we display a select subset of contemporary approaches, which include semantic information into novel algorithms. Third, we highlight approaches that incorporate semantic information into UAV pipelines.\\
In the second section we conduct numerical experiments on the well-defined problem of semantic mapping of a two-dimensional occupancy grid~\cite{gonzalez_unmanned_2016} and show how conventional parameters surrounding the quality of observations, overlap and flight speed influence a naive bayesian filter. We supplement the filter through an informed prior and demonstrate the effectiveness and shortcomings of the approach.\\
As a future prospect, we briefly point to approaches from related semantic fields, such as speech recognition and language processing, which hold the potential to enhance the inclusion of semantic information.\todo{cite lienou - semantic annotation of satellite images using LDA}.
\section{Literature Summary} \label{sec:Lit}
\subsection{Historic Literature} \label{subsec:LitHist}

\subsection{Contemporary Literature}

\subsection{UAV Literature}

% \subsection{Literature on Hierarchical models from NLP}

\section{Methodology} \label{sec:Met}
\subsection{Literature Overview}

\subsection{Experiments}
How does this fit with our previously developed model?

\section{Results} \label{sec:Results}
\subsection{Literature Overview}

\subsection{Simulations}

\section{Discussion} \label{sec:Disc}

\section{Future Work} \label{sec:Fut}
%%%%%%%%%%%%%%%%%%%%%%%%%%%%%%%%%%%%%%%%%%%%%%%%%%%%%%%%%%%%%%%%%%%%%%%%%%%%%%%%%%%%%%%%%%%%%%%%%
\appendices{}              % note there is no {} to put a title. Each appendix has its own title
%%%%%%%%%%%%%%%%%%%%%%%%%%%%%%%%%%%%%%%%%%%%%%%%%%%%%%%%%%%%%%%%%%%%%%%%%%%%%%%%%%%%%%%%%%%%%%%%%
% For a single appendix, use the \appendix{} keyword and do not use the \section command.

\section{More Information}        % first appendix
%%%%%%%%%%%%%%%%%%%%%%%%%%
This is the first appendix. 

\subsection{Comments}
If you have only one appendix, use the ``appendix'' keyword.

\subsection{More Comments}
Use section and subsection keywords as usual.

\section{Yet More Information}    % second appendix
%%%%%%%%%%%%%%%%%%%%%%%%%%%%%%
This is the second appendix.



%%%%%%%%%%%%%%%%%%%%%%%%%%%%%%%%%%%%%%%%%%%%%%%%%%%%%%%%%%%%%%%%%%%%%%%%%%%%%%%%%%%%%%%%%%%%%%%%%%%%%%
\acknowledgments
The authors thank the Office of Naval Research for funding this project.



%%%%%%%%%%%%%%%%%%%%%%%%%%%%%%%%%%%%%%%%%%%%%%%%%%%%%%%%%%%%%%%%%%%%%%%%%%%%%%%%%%%%%%%%%%%%%%%%%%%%%%
\bibliographystyle{IEEEtran}
\bibliography{references}


%%%%%%%%%%%%%%%%%%%%%%%%%%%%%%%%%%%%%%%%%%%%%%%%%%%%%%%%%%%%%%%%%%%%%%%%%%%%%%%%%%%%%%%%%%%%%%%%%%%%%%
\thebiography
%% This biostyle allows you to insert your photo size 1in X 1.25in
\begin{biographywithpic}{Nicolas Mandel}{blankpic.eps}
Blablabla
\end{biographywithpic} 

\begin{biographywithpic}{Michael}{blankpic.eps}
Blablabla
\end{biographywithpic}




\end{document}
